\documentclass[a4paper]{article}

\usepackage{amsmath,amsfonts}

\setlength{\oddsidemargin}{0in}
\setlength{\textwidth}{6.5in}
\setlength{\topmargin}{0in}
\setlength{\headheight}{0in}
\setlength{\headsep}{0in}
\setlength{\textheight}{8.7in}
\usepackage[english]{babel}
\usepackage[utf8x]{inputenc}
\usepackage{amsmath}
\usepackage{graphicx}
\usepackage[colorinlistoftodos]{todonotes}
\usepackage{subfigure}
\usepackage{epstopdf}
\usepackage{float}
\usepackage{listings}
\usepackage{enumerate}

\title{EE5801 Assignment 2}
\author{Wang Ruilin A0260074M}

\begin{document}
\maketitle
\section{Homework list}

\begin{itemize}

\item Tutorial 3

\item Tutorial 4

\end{itemize}

\section{Solution}
\subsection{Tutorial 3}

\begin{enumerate}[Q1]

\item[1]
We give a proof by induction. 
Let $S(n)=1+5+9+...+(4n-3)$, where n is a positive integer.
We want to prove that for every n, $S(n)=2n^2-n$.


Basis step:
$S(1)=2 \times 1^2-1=1$, which is same with sum of 1. 


Inductive step:
Assume $S(k)=2k^2-k$. We want to show $S(k+1)=2(k+1)^2-k$. 
$$S(k+1)=1+5+9+...+(4k-3)+(4(k+1)-3)=S(k)+4(k+1)-3$$
$$S(k+1)=2k^2-k+4(k+1)-3=2k^2+4k+2-1-k$$
$$2k^2+4k+2-1-k=2(k^2+2k+1)-1-k=2(k+1)^2-(k+1)$$


So, we have shown that if $S(k)=2k^2-k$, then $S(k+1)=2(k+1)^2-k$. Since the statement is also true for the basis case, $S(n)=2n^2-n$ for every positive integer n.







\end{enumerate}
\end{document}
